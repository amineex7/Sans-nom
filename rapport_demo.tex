%!TEX encoding = UTF-8 Unicode

%
% Exemple de rapport
% par Pierre Tremblay, Universite Laval
% modifi� par Christian Gagne, Universite Laval
% 14/01/2011 - version 1.3
% modifi� par Robert Bergevin, Universit� Laval
% 24/11/2011
% modifi� par Jean-Yves Chouinard, Universit� Laval
% 11/01/2016
% modifi� par Jean-Yves Chouinard, Universit� Laval
% 04/01/2017
%

%
% Modele d'organisation d'un projet LaTeX 
% rapport/      dossier racine et fichier principal
% rapport/fig   fichiers des figures
% rapport/tex   autres fichiers .tex
%

% ** Preambule **
%
% Ajouter les options au besoin :
%    - "ULlof" pour inclure la liste des figures, requis si "\begin{figure}" utilise
%    - "ULlot" pour inclure la liste des tableaux, requis si "\begin{table}" utilise
%
\documentclass[12pt,ULlot]{./ULrapport}

% Chargement des packages supplementaires (si absent de la classe)
\usepackage[utf8]{inputenc}
\usepackage[table]{xcolor}

% Définition des couleurs pour les tableaux
\definecolor{headerblue}{RGB}{41, 65, 114}
\definecolor{lightgray}{RGB}{240, 240, 240}
\definecolor{mediumgray}{RGB}{220, 220, 220}
\usepackage[autolanguage]{numprint}
\usepackage{icomma}
\usepackage{float}
\usepackage{comment}

%\usepackage[options]{nom_du_package}

% Numeroter les sections en 1, 2, 3... meme sans chapitres
\renewcommand{\thesection}{\arabic{section}}

% Definition d'une commande pour presenter des cellules multilignes dans un tableau
\newcommand{\cellulemultiligne}[1]{\begin{tabular}{@{}c@{}}#1\end{tabular}}

% Definition de colonnes en mode paragraphe avec alignement ajustable
% Cette definition requiert le chargement du package "array"
%    - alignement horizontal, parametre #1 : - \raggedright (aligne a gauche)
%                                            - \centering (centre)
%                                            - \raggedleft (aligne a droite)
%    - alignement vertical, parametre #2 : - p (aligne en haut)
%                                          - m (centre)
%                                          - b (aligne en bas)
%    - largeur, parametre #3 : longueur
\newcolumntype{Z}[3]{>{#1\hspace{0pt}\arraybackslash}#2{#3}}

% Definitions des parametres de la page titre
\TitreProjet{GROUP D'ACHAT YAKABIO}                         % Titre du projet
\TitreRapport{Rapport de projet -- version 1}                       % Titre du rapport
\Destinataire{M. Jean-François Couturier}         % Nom(s) du destinataire
\NumeroEquipe{05}                                     % Numero de l'equipe                            % Nom de l'equipe
\TableauMembres{%                                     % Tableau des membres de l'equipe
   537\,115\,233  & Salma Mouhsin \\\hline
   537\,393\,767  & Vincent Méroz \\\hline
   537\,294\,570  & Mohamed Amine Annane \\\hline
   537\,308\,338  & Iréti Rohan Yannis Kurtis Wilfried Zoumenou \\\hline
}
\DateRemise{13 février 2026}                           % Date de remise du rapport
% Corps du document

\begin{document}

%   Chapitres
%!TEX encoding = UTF-8 Unicode

%
% Chapitre "Introduction"
%

\section{Introduction}
\label{s:intro}

Ce rapport s’inscrit dans la réflexion menée par le groupe d’achat Yakabio afin de maintenir la continuité et la pérennité de ses opérations après des changements importants dans son fonctionnement. À la suite du retrait de personnes clés et de la réorganisation du travail en équipes bénévoles, le groupe fait face à plusieurs enjeux de coordination et de gestion. Dans ce contexte, Mme Karine Lepage a mandaté une équipe d’analyse afin d’évaluer la situation actuelle et de proposer une solution organisationnelle et informatique adaptée. 
\subsection{Introduction au rapport}
Ce rapport présente la charte du projet d’informatisation et d’optimisation des opérations du groupe d’achat Yakabio qui va nous permettre de bien cerner le contexte, les enjeux, les critères de succès, ainsi que les solutions proposée, tout en tenant compte des contraintes, des risques et des impacts liés à la réalization du projet. 




\section{Charte de projet}
\label{sec:charte}

\begin{table}[h]
\centering
\renewcommand{\arraystretch}{1.3}
\begin{tabularx}{\textwidth}{|>{\centering\arraybackslash\columncolor{gray!20}}X|>{\centering\arraybackslash}X|}
\hline
\rule{0pt}{3.2ex}\textbf{Nom du Projet} & \rule{0pt}{3.2ex}Projet d’informatisation et d’optimisation des opérations du groupe d’achat Yakabio \\
\hline
\rule{0pt}{3.2ex}\textbf{Chargé de projet} & \rule{0pt}{3.2ex}Salma Mouhsin  \\ 
\hline
\rule{0pt}{3.2ex}\textbf{Client du projet} & \rule{0pt}{3.2ex}Mme Karine Lepage, coordonnatrice du groupe d'achat Yakabio \\
\hline
\end{tabularx}
\end{table}

\subsection{Mise en contexte de la demande}

Le groupe d'achat Yakabio est affilié à la coopérative BonProBio, située près de Lévis, qui approvisionne en produits biologiques les magasins de produits naturels des régions de Québec et de la Beauce. Yakabio rassemble actuellement 72 membres d'un quartier de la ville de Québec, qui se regroupent pour passer des commandes collectives et ainsi bénéficier des prix de gros offerts par la coopérative.

Depuis sa création, le groupe fonctionnait efficacement grâce à l'implication de M. Jean Després, un retraité bénévole qui assumait à lui seul l'ensemble des tâches de coordination : recevoir les commandes individuelles, compiler et transmettre la commande groupée à BonProBio, réceptionner et distribuer les colis, gérer les paiements et communiquer avec les membres. En 2024, M. Després a reçu un diagnostic de leucémie, ce qui l'a amené à envisager un retrait progressif de ses responsabilités. \newpage

En février 2025, il a définitivement quitté son rôle. Mme Karine Lepage a alors pris le relais en mettant en place une nouvelle organisation reposant sur cinq équipes de bénévoles : Commande, Livraison, Distribution, Paiement et Coordination. Malgré cette réorganisation, des difficultés sont rapidement apparues dès la rentrée de septembre 2025, notamment des désistements de bénévoles et des problèmes de coordination entre les équipes.

Face à ces défis, Mme Lepage souhaite mener une étude approfondie du fonctionnement actuel et mettre en place un système informatisé pour optimiser les opérations. Elle envisage également, à plus long terme, de commercialiser ce logiciel auprès d'autres groupes d'achat afin de rentabiliser l'investissement.

\subsection{Description de la situation actuelle, des besoins et des problématiques}

\subsubsection*{Situation actuelle}

Le groupe d'achat Yakabio fonctionne selon une organisation en cinq équipes de bénévoles. Chaque mois, les membres soumettent leurs commandes individuelles, qui sont compilées puis transmises à la coopérative BonProBio. Les produits sont livrés au centre communautaire CommunoCentre, où ils sont réceptionnés, triés et distribués aux membres. Les paiements se font par chèque : la coordonnatrice dépose les chèques à la caisse populaire et émet ensuite un chèque global à BonProBio.

L'ensemble des processus repose largement sur des tâches manuelles, des documents papier et des communications par courriel ou téléphone. La coordination des bénévoles est assurée principalement par Mme Lepage, qui détient à elle seule une vision globale du fonctionnement du groupe.

\subsubsection*{Problématiques}

\begin{itemize}
    \item \textbf{Désistement des responsables d'équipes :} Il est difficile d'identifier des personnes capables d'assumer la coordination. Lorsqu'un responsable quitte, le remplacement et la formation prennent beaucoup de temps.
    
    \item \textbf{Désistement des membres d'équipes :} Les bénévoles ont des contraintes familiales et professionnelles qui limitent leur disponibilité. Les désistements surviennent souvent à la dernière minute, sans préavis.
    
    \item \textbf{Manque d'organisation des coordonnateurs d'équipes :} Plusieurs responsables ne planifient pas assez tôt et ne contactent pas les membres de leur équipe à temps, ce qui entraîne des problèmes de dernière minute.
    
    \item \textbf{Usage irrégulier du courriel :} Plusieurs membres ne consultent pas régulièrement leur courrier électronique, ce qui complique la communication et la coordination.
    
    \item \textbf{Concentration des connaissances :} Mme Lepage est la seule à avoir une compréhension globale du fonctionnement du groupe. En cas d'absence, le groupe serait fortement fragilisé.
    
    \item \textbf{Processus manuels et sources d'erreurs :} La répartition des frais de livraison, la vérification des bordereaux de commande et la gestion des paiements sont effectuées manuellement, ce qui accroît le risque d'erreurs.
    
    \item \textbf{Manque d'outils de suivi et de rappel :} Les bénévoles ne disposent pas d'outils pour rappeler les tâches et les échéances, ce qui contribue aux oublis et aux désistements.
\end{itemize}

\subsubsection*{Besoins}

\begin{itemize}
    \item \textbf{Automatisation des commandes :} Un système permettant aux membres de soumettre leurs commandes en ligne et de générer automatiquement la commande groupée.
    
    \item \textbf{Gestion automatisée de la facturation :} Calcul automatique des montants, des taxes et des frais de livraison pour chaque membre.
    
    \item \textbf{Suivi des livraisons :} Un module de suivi permettant de savoir quels colis ont été reçus et distribués.
    
    \item \textbf{Coordination des équipes :} Un outil permettant de gérer les disponibilités, d'assigner les tâches et d'envoyer des rappels automatiques.
    
    \item \textbf{Gestion du membership :} Une base de données centralisée avec les coordonnées, l'appartenance aux équipes et l'historique de participation des membres.
    
    \item \textbf{Support multi-fournisseurs :} La possibilité de gérer des commandes auprès de plusieurs fournisseurs avec des calendriers distincts.
    
    \item \textbf{Simplicité d'utilisation :} Une interface simple et intuitive, adaptée à des utilisateurs ayant peu d'habiletés informatiques.
    
    \item \textbf{Traçabilité et visibilité :} Des rapports permettant de suivre les volumes, les coûts et les habitudes de consommation afin de faciliter la prise de décision.
\end{itemize}

\subsection{Critères de succès}

Les critères suivants permettront de déterminer si le projet est un succès à sa conclusion :

\begin{enumerate}
    \item \textbf{Gestion des commandes opérationnelle :} Le système permet aux membres de soumettre leurs commandes en ligne et génère automatiquement la commande groupée à transmettre au fournisseur, sans compilation manuelle.
    
    \item \textbf{Automatisation de la facturation :} Le système calcule automatiquement les montants, les taxes applicables et la répartition des frais de livraison, ce qui réduit les calculs manuels et les erreurs.
    
    \item \textbf{Coordination des équipes facilitée :} Le système permet de gérer les disponibilités, d'assigner les tâches par cycle de commande et d'envoyer des rappels automatiques, ce qui réduit les désistements de dernière minute.
    
    \item \textbf{Gestion centralisée du membership :} Les informations des membres (coordonnées, appartenance aux équipes, cotisations, historique de participation) sont centralisées dans une base de données accessible et à jour.
    
    \item \textbf{Support multi-fournisseurs :} Le système permet de gérer des commandes auprès de plusieurs fournisseurs avec des catalogues et des calendriers distincts.
    
    \item \textbf{Traçabilité complète :} Le système offre une visibilité sur l'historique des commandes, les volumes, les coûts et les habitudes de consommation.
    
    \item \textbf{Facilité d'utilisation :} Les utilisateurs peuvent utiliser le système sans formation extensive. L'interface est intuitive et adaptée à des personnes ayant peu d'habiletés informatiques.
    
    \item \textbf{Déploiement dans les délais :} Le système est livré et opérationnel dans un délai de 18 mois, conformément aux attentes de Mme Lepage.
    
    \item \textbf{Paramétrable et réutilisable :} Le logiciel est suffisamment générique pour être configuré et utilisé par d'autres groupes d'achat, en vue d'une éventuelle commercialisation.
    
    \item \textbf{Réduction de la charge administrative :} Les responsables d'équipes et la coordonnatrice constatent une diminution significative du temps consacré aux tâches administratives.
    
    \item \textbf{Autonomie du groupe :} Le groupe peut fonctionner de manière pérenne sans dépendre d'une seule personne détenant toutes les connaissances.
\end{enumerate}

\subsection{Hypothèses, contraintes et risques}

\subsubsection*{Hypothèses}

\begin{itemize}
    \item Les membres disposent d'un accès à Internet et peuvent utiliser un navigateur web ou une application simple.
    
    \item Mme Lepage et les responsables d'équipes seront disponibles pour l'analyse des besoins, la validation et les tests.
    
    \item Les processus actuels documentés par Mme Lepage (Annexe 2) sont suffisamment stables pour servir de base à l'informatisation.
    
    \item BonProBio maintiendra son mode de fonctionnement actuel (formats des catalogues, bordereaux de commande et livraisons) pendant la durée du projet.
    
    \item Un super-utilisateur sera formé pour former ensuite les autres bénévoles (approche « train the trainer »).
    
    \item Les cas limites et exceptions rares pourront être traités manuellement, sans automatisation complète.
    
    \item Mme Lepage acquérera le matériel et les licences nécessaires dès que l'équipe de projet en fera la demande.
\end{itemize}

\subsubsection*{Contraintes}

\begin{itemize}
    \item \textbf{Contrainte budgétaire :} Le groupe d'achat dispose de ressources financières limitées. Le projet doit réduire au minimum les coûts d'acquisition de logiciels et de matériel.
    
    \item \textbf{Contrainte temporelle :} Mme Lepage souhaite un déploiement dans un délai de 18 mois.
    
    \item \textbf{Contrainte technique - Plateforme :} Le logiciel doit fonctionner sous Windows et nécessiter le moins de logiciels supplémentaires possible (logiciels bureautiques et courriel standards).
    
    \item \textbf{Contrainte technique - Infrastructure :} Le groupe ne dispose pas de ressources informatiques propres. Les utilisateurs emploieront leurs ordinateurs personnels.
    
    \item \textbf{Contrainte d'ergonomie :} L'interface doit être très simple et adaptée à des utilisateurs ayant peu d'habiletés informatiques.
    
    \item \textbf{Contrainte d'installation :} Les procédures d'installation doivent être très simples.
    
    \item \textbf{Contrainte de formation :} La gestion du changement, la documentation et la formation seront prises en charge par Mme Lepage. L'équipe de projet doit seulement former un super-utilisateur.
    
    \item \textbf{Contrainte de connectivité :} Tous les utilisateurs disposent d'une connexion Internet, ce qui est un prérequis pour l'utilisation du système.
\end{itemize}

\subsubsection*{Risques}

\begin{table}[H]
\centering
\scriptsize
\setlength{\tabcolsep}{5pt}
\renewcommand{\arraystretch}{1.543}
\begin{tabularx}{\textwidth}{|p{3.2cm}|c|c|X|}
\hline
\textbf{Risque} & \textbf{Probabilité} & \textbf{Impact} & \textbf{Mesure d'atténuation} \\
\hline
Résistance au changement des bénévoles face au nouveau système & Moyenne & Élevé & Impliquer les utilisateurs dès la conception, offrir une formation adaptée et un accompagnement \\
\hline
Faible adoption du système par les membres peu familiers avec l'informatique & Élevée & Élevé & Concevoir une interface très simple et intuitive, prévoir un mode d'emploi visuel \\
\hline
Indisponibilité de Mme Lepage pour valider les livrables & Moyenne & Élevé & Identifier un remplaçant pour les validations, planifier les rencontres à l'avance \\
\hline
Changement des processus de BonProBio en cours de projet & Faible & Moyen & Concevoir un système flexible et paramétrable \\
\hline
Dépassement de l'échéancier de 18 mois & Moyenne & Moyen & Planification rigoureuse, suivi régulier de l'avancement, priorisation des fonctionnalités essentielles \\
\hline
Problèmes techniques liés à la diversité des équipements des utilisateurs & Moyenne & Moyen & Privilégier une solution web accessible depuis n'importe quel navigateur \\
\hline
Perte de données ou problèmes de sécurité & Faible & Élevé & Mettre en place des sauvegardes régulières et des mesures de sécurité appropriées \\
\hline
\end{tabularx}
\caption{Registre des risques du projet SGI-Yakabio}
\end{table}

\subsection{Solution proposée et portée du projet}

\subsubsection*{Solution proposée}

La solution proposée consiste à développer un système informatisé de gestion intégré pour le groupe d'achat Yakabio, que nous appellerons SGI-Yakabio. Il s'agira d'une application web accessible depuis un navigateur standard, qui permettra aux membres et aux responsables d'équipes d'accomplir leurs tâches de façon centralisée et automatisée.

Le système sera conçu de manière modulaire et paramétrable, de sorte qu'il puisse être adapté et éventuellement commercialisé auprès d'autres groupes d'achat. Il fonctionnera sous Windows et ne nécessitera que des logiciels courants (navigateur web et client courriel).

\subsubsection*{Portée du projet}

Le projet comprend les éléments suivants :

\textbf{Module de gestion des membres :}
\begin{itemize}
    \item Inscription et mise à jour des informations des membres (nom, adresse, téléphone, courriel)
    \item Gestion des cotisations annuelles
    \item Attribution des membres aux différentes équipes
    \item Génération et envoi des cartes de membre
\end{itemize}

\textbf{Module de gestion des fournisseurs et des catalogues :}
\begin{itemize}
    \item Enregistrement des fournisseurs avec leurs coordonnées
    \item Importation et gestion des catalogues de produits (code, description, prix, taxes)
    \item Prise en charge de plusieurs fournisseurs avec des calendriers distincts
\end{itemize}

\textbf{Module de gestion des commandes :}
\begin{itemize}
    \item Interface permettant aux membres de saisir leurs commandes individuelles
    \item Compilation automatique de la commande groupée
    \item Génération des bordereaux de commande dans le format exigé par les fournisseurs
    \item Calcul automatique des montants, des taxes et des totaux
    \item Suivi de l'état des commandes
\end{itemize}

\textbf{Module de gestion des livraisons :}
\begin{itemize}
    \item Enregistrement des livraisons reçues
    \item Calcul et répartition automatique des frais de livraison entre les membres
    \item Génération des factures individuelles pour chaque membre
    \item Suivi de la distribution des colis
\end{itemize}

\textbf{Module de gestion des paiements :}
\begin{itemize}
    \item Enregistrement des paiements reçus des membres
    \item Suivi des chèques et des dépôts bancaires
    \item Génération du sommaire de commande et du chèque destiné au fournisseur
    \item Historique complet des transactions financières
\end{itemize}
\newpage

\textbf{Module de coordination des équipes :}
\begin{itemize}
    \item Gestion des équipes et de leurs membres
    \item Saisie des disponibilités des bénévoles
    \item Attribution des tâches pour chaque cycle de commande
    \item Envoi de rappels automatiques par courriel
    \item Registre de participation des membres aux différentes activités
\end{itemize}

\textbf{Module de paramétrage :}
\begin{itemize}
    \item Configuration des informations du groupe d'achat (nom, adresse, coordonnées)
    \item Personnalisation des documents générés (en-têtes, logos)
    \item Gestion du calendrier des commandes
\end{itemize}

\textbf{Module de rapports et de statistiques :}
\begin{itemize}
    \item Rapports sur les volumes de commandes
    \item Statistiques sur la participation des membres
    \item Historique des commandes par membre
    \item Tableau de bord destiné à la coordonnatrice
\end{itemize}

\textbf{Livrables du projet :}
\begin{itemize}
    \item Application web fonctionnelle SGI-Yakabio
    \item Documentation technique du système
    \item Guide d'utilisation destiné aux utilisateurs
    \item Formation d'un super-utilisateur
    \item Code source et scripts d'installation
\end{itemize}

\subsection{Éléments non inclus dans le projet}

Les éléments suivants ne font pas partie de la portée du projet actuel et ne seront pas développés dans cette première version :

\begin{itemize}
    \item \textbf{Application mobile :} Aucune application mobile n'est prévue. La solution sera accessible uniquement via un navigateur web standard.

    \item \textbf{Intégration avec d'autres fournisseurs que BonProBio :} Dans cette première version, seule l'intégration avec BonProBio sera mise en place. L'ajout d'autres fournisseurs pourra être envisagé dans une version ultérieure du système.

    \item \textbf{Module de gestion financière avancée :} Les fonctionnalités de comptabilité complète et de génération de rapports fiscaux ne sont pas incluses. Le système se limitera au suivi des paiements et à la génération des factures.

    \item \textbf{Paiement en ligne :} L'intégration d'un système de paiement en ligne (carte de crédit, virement bancaire) n'est pas prévue. Les paiements continueront à se faire par chèque.

    \item \textbf{Support multilingue :} L'interface sera offerte uniquement en français.
\end{itemize}

%!TEX encoding = UTF-8 Unicode

%
% Sections "Impacts", "Bénéfices" et "Intervenants"
%

\subsection{Impacts du projet}

La mise en place du système SGI-Yakabio aura plusieurs impacts sur le fonctionnement du groupe d'achat et sur ses membres.

\subsubsection*{Impacts organisationnels}

\begin{itemize}
    \item \textbf{Changement des habitudes de travail :} Les bénévoles devront s'adapter à de nouvelles façons de faire. Les tâches qui étaient auparavant effectuées manuellement seront désormais réalisées à l'aide du système informatique.

    \item \textbf{Nouvelle répartition des responsabilités :} Le système permettra une meilleure distribution des tâches entre les équipes, ce qui réduira la charge de travail de la coordonnatrice et des responsables d'équipes.

    \item \textbf{Besoin de formation :} Les bénévoles devront être formés à l'utilisation du nouveau système. Un effort d'accompagnement sera nécessaire, surtout pour les membres moins à l'aise avec l'informatique.
\end{itemize}

\subsubsection*{Impacts techniques}

\begin{itemize}
    \item \textbf{Dépendance à la technologie :} Le groupe deviendra dépendant du bon fonctionnement du système informatique et de la connexion Internet. Des procédures de secours devront être prévues en cas de panne.

    \item \textbf{Gestion des données :} Les informations des membres et les historiques de commandes seront stockés de façon centralisée, ce qui nécessitera une attention particulière à la sécurité et à la sauvegarde des données.
\end{itemize}

\subsubsection*{Impacts sur les membres}

\begin{itemize}
    \item \textbf{Adaptation au nouveau processus de commande :} Les membres devront apprendre à soumettre leurs commandes en ligne plutôt que par les moyens traditionnels (papier, courriel).

    \item \textbf{Meilleure visibilité :} Les membres auront un accès plus facile à l'information concernant leurs commandes, leurs paiements et leur historique.
\end{itemize}

\subsection{Bénéfices attendus}

Le projet SGI-Yakabio apportera de nombreux bénéfices au groupe d'achat, tant sur le plan opérationnel que sur le plan humain.

\subsubsection*{Bénéfices opérationnels}

\begin{itemize}
    \item \textbf{Réduction des erreurs :} L'automatisation des calculs (montants, taxes, frais de livraison) éliminera les erreurs de calcul qui surviennent lors du traitement manuel des commandes.

    \item \textbf{Gain de temps :} La compilation automatique des commandes et la génération des documents réduiront considérablement le temps consacré aux tâches administratives répétitives.

    \item \textbf{Meilleure coordination :} Les rappels automatiques et la gestion centralisée des disponibilités faciliteront la coordination entre les équipes et réduiront les désistements de dernière minute.

    \item \textbf{Traçabilité améliorée :} L'historique des commandes et des paiements sera facilement accessible, ce qui simplifiera le suivi et la résolution des problèmes.
\end{itemize}

\subsubsection*{Bénéfices pour les membres}

\begin{itemize}
    \item \textbf{Simplicité du processus de commande :} Les membres pourront passer leurs commandes en ligne à tout moment, sans avoir à se déplacer ou à envoyer des courriels.

    \item \textbf{Transparence :} Chaque membre pourra consulter l'état de ses commandes, ses factures et son historique de participation.

    \item \textbf{Réduction des délais :} Le traitement plus rapide des commandes permettra une meilleure planification et une distribution plus efficace des produits.
\end{itemize}

\subsubsection*{Bénéfices pour la pérennité du groupe}

\begin{itemize}
    \item \textbf{Réduction de la dépendance aux personnes clés :} La documentation des processus dans le système et la centralisation de l'information permettront au groupe de fonctionner même en cas d'absence de la coordonnatrice.

    \item \textbf{Facilité d'intégration des nouveaux bénévoles :} Les nouveaux membres pourront se familiariser plus rapidement avec les opérations grâce à une interface claire et des procédures bien définies.

    \item \textbf{Potentiel de croissance :} Le système pourra accommoder une augmentation du nombre de membres ou l'ajout de nouveaux fournisseurs dans le futur.

    \item \textbf{Possibilité de commercialisation :} Le logiciel pourra être adapté et vendu à d'autres groupes d'achat, ce qui permettrait de rentabiliser l'investissement initial.
\end{itemize}

\subsection{Intervenants du projet}

Cette section présente les différentes personnes et entités impliquées dans le projet, ainsi que leur rôle respectif.

\subsubsection*{Client du projet}

\begin{itemize}
    \item \textbf{Mme Karine Lepage} -- Coordonnatrice du groupe d'achat Yakabio. Elle est la principale interlocutrice pour la définition des besoins, la validation des livrables et l'acceptation finale du système. Elle sera également responsable de la formation des bénévoles et de la gestion du changement au sein du groupe.
\end{itemize}

\subsubsection*{Utilisateurs du système}

\begin{itemize}
    \item \textbf{La coordonnatrice (Mme Lepage)} -- Utilisatrice principale du système avec des droits d'administration complets. Elle aura accès à toutes les fonctionnalités, y compris les rapports et les statistiques.

    \item \textbf{Les responsables d'équipes} -- Utilisateurs ayant des droits de gestion pour leur équipe respective. Ils pourront gérer les disponibilités, assigner les tâches et consulter les informations pertinentes à leur équipe.

    \item \textbf{Les bénévoles} -- Utilisateurs ayant des droits limités leur permettant de consulter les tâches qui leur sont assignées et de confirmer leur disponibilité.

    \item \textbf{Les membres du groupe d'achat} -- Utilisateurs ayant des droits de base pour passer des commandes, consulter leurs factures et mettre à jour leurs informations personnelles.
\end{itemize}

\subsubsection*{Parties prenantes externes}

\begin{itemize}
    \item \textbf{Coopérative BonProBio} -- Fournisseur principal du groupe d'achat. Bien qu'elle ne soit pas directement impliquée dans le projet, le système devra être compatible avec ses formats de catalogues et de bordereaux de commande.

    \item \textbf{Centre communautaire CommunoCentre} -- Lieu de réception et de distribution des commandes. Le système devra tenir compte des contraintes liées à ce lieu (horaires, disponibilité des locaux).
\end{itemize}

%!TEX encoding = UTF-8 Unicode

%
% Section "Conclusion"
%

\section{Conclusion}
\label{s:conclusion}

Ce rapport nous a permis d'analyser la situation actuelle du groupe d'achat Yakabio dans un contexte de changements organisationnels importants. L'étude du fonctionnement du groupe a mis en évidence plusieurs faiblesses, notamment la forte dépendance à une seule personne, la difficulté de coordination entre les bénévoles et la multiplication des erreurs administratives liées à des processus majoritairement manuels.

La rédaction de la charte de projet a permis de structurer la réflexion autour de ces enjeux en identifiant clairement les problématiques, les besoins, les critères de succès ainsi que les risques et contraintes associés au projet. Cette démarche constitue une étape essentielle afin de poser des bases solides pour la mise en place d'une solution organisationnelle et informatique adaptée aux réalités de Yakabio.

\subsection{Bilan du travail réalisé et faisabilité du projet}

À ce stade du projet, plusieurs éléments clés ont été complétés, notamment l'analyse du contexte, l'identification des problématiques et des besoins, la définition des critères de succès ainsi que la rédaction du compte rendu de réunion. La répartition des tâches au sein de l'équipe a également été effectuée, ce qui permet une progression structurée et organisée du travail.

Du point de vue de la faisabilité, le projet apparaît réaliste et réalisable. Les besoins exprimés par le groupe Yakabio sont clairs et bien définis, et la solution repose sur des outils informatiques simples et accessibles, adaptés à un environnement bénévole. Aucun obstacle majeur n'a été identifié à ce stade. Bien que certaines contraintes et risques existent, notamment liés à la disponibilité des bénévoles et à la gestion du changement, ceux-ci demeurent gérables dans le cadre du projet proposé. Le projet est donc jugé faisable tant sur le plan organisationnel que technique.

\subsection{Recommandations pour la suite du travail}

Afin d'assurer la réussite du projet du groupe d'achat Yakabio, il est recommandé :

\begin{itemize}
    \item D'assurer un suivi de l'échéancier et des responsabilités, compte tenu de la disponibilité des bénévoles et du caractère communautaire de l'organisation.

    \item De valider régulièrement les livrables avec la coordonnatrice du groupe, afin de s'assurer que les besoins exprimés sont bien compris et que les solutions proposées demeurent alignées avec la réalité opérationnelle de Yakabio.

    \item De documenter clairement les processus et les décisions prises, dans le but de réduire la dépendance envers des personnes clés et de faciliter la continuité des opérations lors de changements futurs.

    \item D'impliquer progressivement les bénévoles dans le projet, afin de favoriser l'adhésion aux changements.

    \item De prévoir un plan de formation simple et accessible, permettant aux bénévoles de se familiariser avec les nouveaux outils ou processus sans alourdir leur charge de travail.

    \item De maintenir une communication claire et régulière au sein de l'équipe de projet, en utilisant des outils collaboratifs pour assurer une coordination efficace et un suivi constant de l'avancement.
\end{itemize}

%!TEX encoding = UTF-8 Unicode

%
% Section "Gestion de projets"
%

\section{Compte rendu de la gestion de projet}
\label{s:gestion}

Cette section présente l'organisation de notre équipe de travail, les activités réalisées dans le cadre de ce projet, ainsi qu'une réflexion sur notre fonctionnement en tant qu'équipe.

\subsection{Composition de l'équipe et rôles}

L'équipe chargée du projet est constituée de quatre membres. Le tableau suivant présente chacun d'eux ainsi que les principales responsabilités qui leur sont attribuées au sein du groupe.

\begin{table}[H]
\centering
\renewcommand{\arraystretch}{1.4}
\begin{tabularx}{\textwidth}{|l|X|}
\hline
\rowcolor{headerblue}
\textcolor{white}{\textbf{Nom}} & \textcolor{white}{\textbf{Description des tâches}} \\
\hline
\rowcolor{lightgray}
Salma Mouhsin & Chargée de projet, coordination de l'équipe, communication avec le client et rédaction du rapport \\
\hline
Vincent Méroz & Analyse des besoins, rédaction des sections sur les problématiques et critères de succès \\
\hline
\rowcolor{lightgray}
Mohamed Amine Annane & Analyse de la solution, rédaction des sections sur les impacts et la portée du projet \\
\hline
Iréti Rohan Yannis Kurtis Wilfried Zoumenou & Analyse des risques, rédaction des sections sur les hypothèses et la conclusion \\
\hline
\end{tabularx}
\caption{Composition de l'équipe et répartition des tâches}
\end{table}

\subsection{Compte rendu des principales activités réalisées}

\subsubsection*{Réunion 1 -- Discussion du TP et répartition des tâches}

\begin{table}[H]
\centering
\renewcommand{\arraystretch}{1.3}
\begin{tabularx}{\textwidth}{|>{\bfseries}l|X|}
\hline
\cellcolor{headerblue}\textcolor{white}{Date} & 29 janvier 2026 \\
\hline
\rowcolor{lightgray}
Heure & De 19h à 19h30 \\
\hline
\cellcolor{headerblue}\textcolor{white}{Endroit} & Teams (réunion virtuelle) \\
\hline
\rowcolor{lightgray}
Objet & Discussion du TP et répartition des tâches \\
\hline
\cellcolor{headerblue}\textcolor{white}{Participants} & Tous les membres de l'équipe \\
\hline
\end{tabularx}
\end{table}

\textbf{Ordre du jour :}

\begin{table}[H]
\centering
\renewcommand{\arraystretch}{1.3}
\begin{tabularx}{\textwidth}{|X|c|}
\hline
\rowcolor{headerblue}
\textcolor{white}{\textbf{Point à l'ordre du jour}} & \textcolor{white}{\textbf{Durée}} \\
\hline
\rowcolor{lightgray}
1. Ouverture de la réunion & 2 min \\
\hline
2. Présentation du TP1 et des attentes du professeur & 7 min \\
\hline
\rowcolor{lightgray}
3. Discussion sur la structure du rapport et la charte de projet & 10 min \\
\hline
4. Nomination du chargé de projet & 3 min \\
\hline
\rowcolor{lightgray}
5. Planification des prochaines étapes & 8 min \\
\hline
\end{tabularx}
\caption{Ordre du jour de la réunion du 29 janvier 2026}
\end{table}
\newpage

\textbf{Décisions prises lors de cette réunion :}
\begin{itemize}
    \item Salma Mouhsin a été nommée chargée de projet à l'unanimité.
    \item La structure du rapport a été validée selon le modèle fourni par le professeur.
    \item Chaque membre s'est vu attribuer des sections spécifiques à rédiger.
    \item La prochaine réunion a été planifiée pour faire le point sur l'avancement.
\end{itemize}
\begin{comment}
\subsubsection*{Réunion 3 -- Révision finale}

\begin{table}[H]
\centering
\renewcommand{\arraystretch}{1.3}
\begin{tabularx}{\textwidth}{|>{\bfseries}l|X|}
\hline
\cellcolor{headerblue}\textcolor{white}{Date} & 10 février 2026 \\
\hline
\rowcolor{lightgray}
Heure & De 18h à 19h \\
\hline
\cellcolor{headerblue}\textcolor{white}{Endroit} & Teams (réunion virtuelle) \\
\hline
\rowcolor{lightgray}
Objet & Révision finale et validation du rapport \\
\hline
\cellcolor{headerblue}\textcolor{white}{Participants} & Tous les membres de l'équipe \\
\hline
\end{tabularx}
\end{table}

\textbf{Points abordés :}
\begin{itemize}
    \item Relecture complète du rapport par tous les membres.
    \item Correction des erreurs de français et harmonisation du style.
    \item Vérification de la conformité avec le barème d'évaluation.
    \item Validation finale du document avant la remise.
\end{itemize}
\end{comment}
\subsection{Problèmes rencontrés et suggestions de solutions}

Au cours de ce projet, nous avons rencontré quelques difficultés que nous avons su surmonter grâce à une bonne communication au sein de l'équipe.

\subsubsection*{Problèmes rencontrés}

\begin{itemize}
    \item \textbf{Coordination des horaires :} Il a parfois été difficile de trouver des créneaux où tous les membres étaient disponibles pour les réunions d'équipe, compte tenu des emplois du temps chargés de chacun. Nous avons résolu ce problème en privilégiant les réunions en soirée sur Teams.

    \item \textbf{Compréhension du contexte du groupe d'achat :} Au départ, il n'était pas facile de bien comprendre toutes les subtilités du fonctionnement d'un groupe d'achat communautaire. La lecture attentive des documents fournis et les discussions en équipe nous ont permis de clarifier plusieurs points.

    \item \textbf{Harmonisation du style de rédaction :} Comme chaque membre a rédigé des sections différentes, il a fallu un travail de révision pour assurer une cohérence dans le ton et le style du rapport.

    \item \textbf{Gestion du temps :} Avec les autres cours et obligations, il a fallu bien planifier le travail pour respecter l'échéance de remise.
\end{itemize}

\subsubsection*{Suggestions de solutions}

\begin{itemize}
    \item \textbf{Planifier les rencontres dès le début :} Fixer les dates des réunions dès le lancement du projet pour éviter les conflits d'horaire.

    \item \textbf{Établir un guide de style commun :} Définir dès le départ des conventions de rédaction pour faciliter l'harmonisation des contributions de chaque membre.

    \item \textbf{Utiliser des outils collaboratifs :} Continuer à utiliser Teams et d'autres outils de partage de documents pour faciliter la collaboration à distance.

    \item \textbf{Prévoir des marges de temps :} Intégrer des périodes tampons dans l'échéancier pour faire face aux imprévus.
\end{itemize}

\subsection{Évaluation globale de la gestion de projet et d'équipe}

Dans l'ensemble, notre équipe a bien fonctionné tout au long de ce projet. La communication entre les membres a été bonne, et chacun a respecté ses engagements en termes de délais et de qualité du travail.

\subsubsection*{Points forts de l'équipe}

\begin{itemize}
    \item \textbf{Bonne répartition des tâches :} Dès la première réunion, chaque membre s'est vu attribuer des responsabilités claires, ce qui a évité les chevauchements et les oublis.

    \item \textbf{Communication efficace :} L'utilisation de Teams a permis de maintenir un contact régulier entre les membres, même en dehors des réunions formelles.

    \item \textbf{Entraide :} Lorsqu'un membre avait besoin d'aide ou de clarification, les autres étaient disponibles pour l'appuyer.

    \item \textbf{Respect des échéances :} Toutes les étapes du projet ont été réalisées dans les délais prévus, ce qui nous a permis de remettre le rapport à temps.
\end{itemize}

\subsubsection*{Points à améliorer}

\begin{itemize}
    \item \textbf{Réunions plus fréquentes :} Des points de suivi plus réguliers auraient permis de détecter plus tôt certains problèmes de coordination.

\end{itemize}

\subsubsection*{Conclusion}

Ce projet a été une expérience enrichissante qui nous a permis de développer nos compétences en analyse de besoins, en rédaction technique et en travail collaboratif. La réalisation de la charte de projet nous a donné une meilleure compréhension des enjeux liés à la gestion d'un projet informatique dans un contexte communautaire. Nous sommes satisfaits du résultat obtenu et confiants que ce rapport constitue une base solide pour la suite du projet SGI-Yakabio.

%\input{tex/exemples_LaTeX}

%   Annexes
\appendix


\end{document}
% Fin du document
