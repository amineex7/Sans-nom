%!TEX encoding = UTF-8 Unicode

%
% Exemple de rapport
% par Pierre Tremblay, Universite Laval
% modifi� par Christian Gagne, Universite Laval
% 14/01/2011 - version 1.3
% modifi� par Robert Bergevin, Universit� Laval
% 24/11/2011
% modifi� par Jean-Yves Chouinard, Universit� Laval
% 11/01/2016
% modifi� par Jean-Yves Chouinard, Universit� Laval
% 04/01/2017
%

%
% Modele d'organisation d'un projet LaTeX 
% rapport/      dossier racine et fichier principal
% rapport/fig   fichiers des figures
% rapport/tex   autres fichiers .tex
%

% ** Preambule **
%
% Ajouter les options au besoin :
%    - "ULlof" pour inclure la liste des figures, requis si "\begin{figure}" utilise
%    - "ULlot" pour inclure la liste des tableaux, requis si "\begin{table}" utilise
%
\documentclass[12pt,ULlof,ULlot]{ULrapport}

% Chargement des packages supplementaires (si absent de la classe)
\usepackage[utf8]{inputenc}
\usepackage[table]{xcolor}
\usepackage[autolanguage]{numprint}
\usepackage{icomma}
\usepackage{float}
%\usepackage[options]{nom_du_package}

% Numeroter les sections en 1, 2, 3... meme sans chapitres
\renewcommand{\thesection}{\arabic{section}}

% Definition d'une commande pour presenter des cellules multilignes dans un tableau
\newcommand{\cellulemultiligne}[1]{\begin{tabular}{@{}c@{}}#1\end{tabular}}

% Definition de colonnes en mode paragraphe avec alignement ajustable
% Cette definition requiert le chargement du package "array"
%    - alignement horizontal, parametre #1 : - \raggedright (aligne a gauche)
%                                            - \centering (centre)
%                                            - \raggedleft (aligne a droite)
%    - alignement vertical, parametre #2 : - p (aligne en haut)
%                                          - m (centre)
%                                          - b (aligne en bas)
%    - largeur, parametre #3 : longueur
\newcolumntype{Z}[3]{>{#1\hspace{0pt}\arraybackslash}#2{#3}}

% Definitions des parametres de la page titre
\TitreProjet{GROUP D'ACHAT YAKABIO}                         % Titre du projet
\TitreRapport{Rapport de projet -- version 1}                       % Titre du rapport
\Destinataire{M. Jean-François Couturier}         % Nom(s) du destinataire
\NumeroEquipe{05}                                     % Numero de l'equipe                            % Nom de l'equipe
\TableauMembres{%                                     % Tableau des membres de l'equipe
   111\,111\,111  & Salma Mouhsin   & Responsable de l'équipe\\\hline      % matricule & nom & \\\hline
   222\,222\,222  & Vincent Méroz        & \\\hline        % matricule & nom & \\\hline
   333\,333\,333  & Mohamed Amine Annane         & \\\hline        % matricule & nom & \\\hline
   444\,444\,444  & Iréti Rohan Yannis Kurtis Wilfried Zoumenou     & \\\hline        % matricule & nom & \\\hline
}
\DateRemise{13 février 2026}                           % Date de remise du rapport
% Corps du document

\begin{document}

%   Chapitres
%!TEX encoding = UTF-8 Unicode

%
% Chapitre "Introduction"
%

\section{Introduction}
\label{s:intro}

Ce rapport s’inscrit dans la réflexion menée par le groupe d’achat Yakabio afin de maintenir la continuité et la pérennité de ses opérations après des changements importants dans son fonctionnement. À la suite du retrait de personnes clés et de la réorganisation du travail en équipes bénévoles, le groupe fait face à plusieurs enjeux de coordination et de gestion. Dans ce contexte, Mme Karine Lepage a mandaté une équipe d’analyse afin d’évaluer la situation actuelle et de proposer une solution organisationnelle et informatique adaptée. 
\subsection{Introduction au rapport}
Ce rapport présente la charte du projet d’informatisation et d’optimisation des opérations du groupe d’achat Yakabio qui va nous permettre de bien cerner le contexte, les enjeux, les critères de succès, ainsi que les solutions proposée, tout en tenant compte des contraintes, des risques et des impacts liés à la réalization du projet. 




\section{Charte de projet}
\label{sec:charte}

\begin{table}[h]
\centering
\renewcommand{\arraystretch}{1.3}
\begin{tabularx}{\textwidth}{|>{\centering\arraybackslash\columncolor{gray!20}}X|>{\centering\arraybackslash}X|}
\hline
\rule{0pt}{3.2ex}\textbf{Nom du Projet} & \rule{0pt}{3.2ex}Projet d’informatisation et d’optimisation des opérations du groupe d’achat Yakabio \\
\hline
\rule{0pt}{3.2ex}\textbf{Chargé de projet} & \rule{0pt}{3.2ex}Salma Mouhsin  \\ 
\hline
\rule{0pt}{3.2ex}\textbf{Client du projet} & \rule{0pt}{3.2ex}Mme Karine Lepage, coordonnatrice du groupe d'achat Yakabio \\
\hline
\end{tabularx}
\end{table}

\subsection{Mise en contexte de la demande}

Le groupe d'achat Yakabio est affilié à la coopérative BonProBio, située près de Lévis, qui approvisionne en produits biologiques les magasins de produits naturels des régions de Québec et de la Beauce. Yakabio rassemble actuellement 72 membres d'un quartier de la ville de Québec, qui se regroupent pour passer des commandes collectives et ainsi bénéficier des prix de gros offerts par la coopérative.

Depuis sa création, le groupe fonctionnait efficacement grâce à l'implication de M. Jean Després, un retraité bénévole qui assumait à lui seul l'ensemble des tâches de coordination : recevoir les commandes individuelles, compiler et transmettre la commande groupée à BonProBio, réceptionner et distribuer les colis, gérer les paiements et communiquer avec les membres. En 2024, M. Després a reçu un diagnostic de leucémie, ce qui l'a amené à envisager un retrait progressif de ses responsabilités. \newpage

En février 2025, il a définitivement quitté son rôle. Mme Karine Lepage a alors pris le relais en mettant en place une nouvelle organisation reposant sur cinq équipes de bénévoles : Commande, Livraison, Distribution, Paiement et Coordination. Malgré cette réorganisation, des difficultés sont rapidement apparues dès la rentrée de septembre 2025, notamment des désistements de bénévoles et des problèmes de coordination entre les équipes.

Face à ces défis, Mme Lepage souhaite mener une étude approfondie du fonctionnement actuel et mettre en place un système informatisé pour optimiser les opérations. Elle envisage également, à plus long terme, de commercialiser ce logiciel auprès d'autres groupes d'achat afin de rentabiliser l'investissement.

\subsection{Description de la situation actuelle, des besoins et des problématiques}

\subsubsection*{Situation actuelle}

Le groupe d'achat Yakabio fonctionne selon une organisation en cinq équipes de bénévoles. Chaque mois, les membres soumettent leurs commandes individuelles, qui sont compilées puis transmises à la coopérative BonProBio. Les produits sont livrés au centre communautaire CommunoCentre, où ils sont réceptionnés, triés et distribués aux membres. Les paiements se font par chèque : la coordonnatrice dépose les chèques à la caisse populaire et émet ensuite un chèque global à BonProBio.

L'ensemble des processus repose largement sur des tâches manuelles, des documents papier et des communications par courriel ou téléphone. La coordination des bénévoles est assurée principalement par Mme Lepage, qui détient à elle seule une vision globale du fonctionnement du groupe.

\subsubsection*{Problématiques}

\begin{itemize}
    \item \textbf{Désistement des responsables d'équipes :} Il est difficile d'identifier des personnes capables d'assumer la coordination. Lorsqu'un responsable quitte, le remplacement et la formation prennent beaucoup de temps.
    
    \item \textbf{Désistement des membres d'équipes :} Les bénévoles ont des contraintes familiales et professionnelles qui limitent leur disponibilité. Les désistements surviennent souvent à la dernière minute, sans préavis.
    
    \item \textbf{Manque d'organisation des coordonnateurs d'équipes :} Plusieurs responsables ne planifient pas assez tôt et ne contactent pas les membres de leur équipe à temps, ce qui entraîne des problèmes de dernière minute.
    
    \item \textbf{Usage irrégulier du courriel :} Plusieurs membres ne consultent pas régulièrement leur courrier électronique, ce qui complique la communication et la coordination.
    
    \item \textbf{Concentration des connaissances :} Mme Lepage est la seule à avoir une compréhension globale du fonctionnement du groupe. En cas d'absence, le groupe serait fortement fragilisé.
    
    \item \textbf{Processus manuels et sources d'erreurs :} La répartition des frais de livraison, la vérification des bordereaux de commande et la gestion des paiements sont effectuées manuellement, ce qui accroît le risque d'erreurs.
    
    \item \textbf{Manque d'outils de suivi et de rappel :} Les bénévoles ne disposent pas d'outils pour rappeler les tâches et les échéances, ce qui contribue aux oublis et aux désistements.
\end{itemize}

\subsubsection*{Besoins}

\begin{itemize}
    \item \textbf{Automatisation des commandes :} Un système permettant aux membres de soumettre leurs commandes en ligne et de générer automatiquement la commande groupée.
    
    \item \textbf{Gestion automatisée de la facturation :} Calcul automatique des montants, des taxes et des frais de livraison pour chaque membre.
    
    \item \textbf{Suivi des livraisons :} Un module de suivi permettant de savoir quels colis ont été reçus et distribués.
    
    \item \textbf{Coordination des équipes :} Un outil permettant de gérer les disponibilités, d'assigner les tâches et d'envoyer des rappels automatiques.
    
    \item \textbf{Gestion du membership :} Une base de données centralisée avec les coordonnées, l'appartenance aux équipes et l'historique de participation des membres.
    
    \item \textbf{Support multi-fournisseurs :} La possibilité de gérer des commandes auprès de plusieurs fournisseurs avec des calendriers distincts.
    
    \item \textbf{Simplicité d'utilisation :} Une interface simple et intuitive, adaptée à des utilisateurs ayant peu d'habiletés informatiques.
    
    \item \textbf{Traçabilité et visibilité :} Des rapports permettant de suivre les volumes, les coûts et les habitudes de consommation afin de faciliter la prise de décision.
\end{itemize}

\subsection{Critères de succès}

Les critères suivants permettront de déterminer si le projet est un succès à sa conclusion :

\begin{enumerate}
    \item \textbf{Gestion des commandes opérationnelle :} Le système permet aux membres de soumettre leurs commandes en ligne et génère automatiquement la commande groupée à transmettre au fournisseur, sans compilation manuelle.
    
    \item \textbf{Automatisation de la facturation :} Le système calcule automatiquement les montants, les taxes applicables et la répartition des frais de livraison, ce qui réduit les calculs manuels et les erreurs.
    
    \item \textbf{Coordination des équipes facilitée :} Le système permet de gérer les disponibilités, d'assigner les tâches par cycle de commande et d'envoyer des rappels automatiques, ce qui réduit les désistements de dernière minute.
    
    \item \textbf{Gestion centralisée du membership :} Les informations des membres (coordonnées, appartenance aux équipes, cotisations, historique de participation) sont centralisées dans une base de données accessible et à jour.
    
    \item \textbf{Support multi-fournisseurs :} Le système permet de gérer des commandes auprès de plusieurs fournisseurs avec des catalogues et des calendriers distincts.
    
    \item \textbf{Traçabilité complète :} Le système offre une visibilité sur l'historique des commandes, les volumes, les coûts et les habitudes de consommation.
    
    \item \textbf{Facilité d'utilisation :} Les utilisateurs peuvent utiliser le système sans formation extensive. L'interface est intuitive et adaptée à des personnes ayant peu d'habiletés informatiques.
    
    \item \textbf{Déploiement dans les délais :} Le système est livré et opérationnel dans un délai de 18 mois, conformément aux attentes de Mme Lepage.
    
    \item \textbf{Paramétrable et réutilisable :} Le logiciel est suffisamment générique pour être configuré et utilisé par d'autres groupes d'achat, en vue d'une éventuelle commercialisation.
    
    \item \textbf{Réduction de la charge administrative :} Les responsables d'équipes et la coordonnatrice constatent une diminution significative du temps consacré aux tâches administratives.
    
    \item \textbf{Autonomie du groupe :} Le groupe peut fonctionner de manière pérenne sans dépendre d'une seule personne détenant toutes les connaissances.
\end{enumerate}

\subsection{Hypothèses, contraintes et risques}

\subsubsection*{Hypothèses}

\begin{itemize}
    \item Les membres disposent d'un accès à Internet et peuvent utiliser un navigateur web ou une application simple.
    
    \item Mme Lepage et les responsables d'équipes seront disponibles pour l'analyse des besoins, la validation et les tests.
    
    \item Les processus actuels documentés par Mme Lepage (Annexe 2) sont suffisamment stables pour servir de base à l'informatisation.
    
    \item BonProBio maintiendra son mode de fonctionnement actuel (formats des catalogues, bordereaux de commande et livraisons) pendant la durée du projet.
    
    \item Un super-utilisateur sera formé pour former ensuite les autres bénévoles (approche « train the trainer »).
    
    \item Les cas limites et exceptions rares pourront être traités manuellement, sans automatisation complète.
    
    \item Mme Lepage acquérera le matériel et les licences nécessaires dès que l'équipe de projet en fera la demande.
\end{itemize}

\subsubsection*{Contraintes}

\begin{itemize}
    \item \textbf{Contrainte budgétaire :} Le groupe d'achat dispose de ressources financières limitées. Le projet doit réduire au minimum les coûts d'acquisition de logiciels et de matériel.
    
    \item \textbf{Contrainte temporelle :} Mme Lepage souhaite un déploiement dans un délai de 18 mois.
    
    \item \textbf{Contrainte technique - Plateforme :} Le logiciel doit fonctionner sous Windows et nécessiter le moins de logiciels supplémentaires possible (logiciels bureautiques et courriel standards).
    
    \item \textbf{Contrainte technique - Infrastructure :} Le groupe ne dispose pas de ressources informatiques propres. Les utilisateurs emploieront leurs ordinateurs personnels.
    
    \item \textbf{Contrainte d'ergonomie :} L'interface doit être très simple et adaptée à des utilisateurs ayant peu d'habiletés informatiques.
    
    \item \textbf{Contrainte d'installation :} Les procédures d'installation doivent être très simples.
    
    \item \textbf{Contrainte de formation :} La gestion du changement, la documentation et la formation seront prises en charge par Mme Lepage. L'équipe de projet doit seulement former un super-utilisateur.
    
    \item \textbf{Contrainte de connectivité :} Tous les utilisateurs disposent d'une connexion Internet, ce qui est un prérequis pour l'utilisation du système.
\end{itemize}

\subsubsection*{Risques}

\begin{table}[H]
\centering
\scriptsize
\setlength{\tabcolsep}{5pt}
\renewcommand{\arraystretch}{1.543}
\begin{tabularx}{\textwidth}{|p{3.2cm}|c|c|X|}
\hline
\textbf{Risque} & \textbf{Probabilité} & \textbf{Impact} & \textbf{Mesure d'atténuation} \\
\hline
Résistance au changement des bénévoles face au nouveau système & Moyenne & Élevé & Impliquer les utilisateurs dès la conception, offrir une formation adaptée et un accompagnement \\
\hline
Faible adoption du système par les membres peu familiers avec l'informatique & Élevée & Élevé & Concevoir une interface très simple et intuitive, prévoir un mode d'emploi visuel \\
\hline
Indisponibilité de Mme Lepage pour valider les livrables & Moyenne & Élevé & Identifier un remplaçant pour les validations, planifier les rencontres à l'avance \\
\hline
Changement des processus de BonProBio en cours de projet & Faible & Moyen & Concevoir un système flexible et paramétrable \\
\hline
Dépassement de l'échéancier de 18 mois & Moyenne & Moyen & Planification rigoureuse, suivi régulier de l'avancement, priorisation des fonctionnalités essentielles \\
\hline
Problèmes techniques liés à la diversité des équipements des utilisateurs & Moyenne & Moyen & Privilégier une solution web accessible depuis n'importe quel navigateur \\
\hline
Perte de données ou problèmes de sécurité & Faible & Élevé & Mettre en place des sauvegardes régulières et des mesures de sécurité appropriées \\
\hline
\end{tabularx}
\caption{Registre des risques du projet SGI-Yakabio}
\end{table}

\subsection{Solution proposée et portée du projet}

\subsubsection*{Solution proposée}

La solution proposée consiste à développer un système informatisé de gestion intégré pour le groupe d'achat Yakabio, que nous appellerons SGI-Yakabio. Il s'agira d'une application web accessible depuis un navigateur standard, qui permettra aux membres et aux responsables d'équipes d'accomplir leurs tâches de façon centralisée et automatisée.

Le système sera conçu de manière modulaire et paramétrable, de sorte qu'il puisse être adapté et éventuellement commercialisé auprès d'autres groupes d'achat. Il fonctionnera sous Windows et ne nécessitera que des logiciels courants (navigateur web et client courriel).

\subsubsection*{Portée du projet}

Le projet comprend les éléments suivants :

\textbf{Module de gestion des membres :}
\begin{itemize}
    \item Inscription et mise à jour des informations des membres (nom, adresse, téléphone, courriel)
    \item Gestion des cotisations annuelles
    \item Attribution des membres aux différentes équipes
    \item Génération et envoi des cartes de membre
\end{itemize}

\textbf{Module de gestion des fournisseurs et des catalogues :}
\begin{itemize}
    \item Enregistrement des fournisseurs avec leurs coordonnées
    \item Importation et gestion des catalogues de produits (code, description, prix, taxes)
    \item Prise en charge de plusieurs fournisseurs avec des calendriers distincts
\end{itemize}

\textbf{Module de gestion des commandes :}
\begin{itemize}
    \item Interface permettant aux membres de saisir leurs commandes individuelles
    \item Compilation automatique de la commande groupée
    \item Génération des bordereaux de commande dans le format exigé par les fournisseurs
    \item Calcul automatique des montants, des taxes et des totaux
    \item Suivi de l'état des commandes
\end{itemize}

\textbf{Module de gestion des livraisons :}
\begin{itemize}
    \item Enregistrement des livraisons reçues
    \item Calcul et répartition automatique des frais de livraison entre les membres
    \item Génération des factures individuelles pour chaque membre
    \item Suivi de la distribution des colis
\end{itemize}

\textbf{Module de gestion des paiements :}
\begin{itemize}
    \item Enregistrement des paiements reçus des membres
    \item Suivi des chèques et des dépôts bancaires
    \item Génération du sommaire de commande et du chèque destiné au fournisseur
    \item Historique complet des transactions financières
\end{itemize}
\newpage

\textbf{Module de coordination des équipes :}
\begin{itemize}
    \item Gestion des équipes et de leurs membres
    \item Saisie des disponibilités des bénévoles
    \item Attribution des tâches pour chaque cycle de commande
    \item Envoi de rappels automatiques par courriel
    \item Registre de participation des membres aux différentes activités
\end{itemize}

\textbf{Module de paramétrage :}
\begin{itemize}
    \item Configuration des informations du groupe d'achat (nom, adresse, coordonnées)
    \item Personnalisation des documents générés (en-têtes, logos)
    \item Gestion du calendrier des commandes
\end{itemize}

\textbf{Module de rapports et de statistiques :}
\begin{itemize}
    \item Rapports sur les volumes de commandes
    \item Statistiques sur la participation des membres
    \item Historique des commandes par membre
    \item Tableau de bord destiné à la coordonnatrice
\end{itemize}

\textbf{Livrables du projet :}
\begin{itemize}
    \item Application web fonctionnelle SGI-Yakabio
    \item Documentation technique du système
    \item Guide d'utilisation destiné aux utilisateurs
    \item Formation d'un super-utilisateur
    \item Code source et scripts d'installation
\end{itemize}

\subsection{Éléments non inclus dans le projet}

Les éléments suivants ne font pas partie de la portée du projet actuel et ne seront pas développés dans cette première version :

\begin{itemize}
    \item \textbf{Application mobile :} Aucune application mobile n'est prévue. La solution sera accessible uniquement via un navigateur web standard.

    \item \textbf{Intégration avec d'autres fournisseurs que BonProBio :} Dans cette première version, seule l'intégration avec BonProBio sera mise en place. L'ajout d'autres fournisseurs pourra être envisagé dans une version ultérieure du système.

    \item \textbf{Module de gestion financière avancée :} Les fonctionnalités de comptabilité complète et de génération de rapports fiscaux ne sont pas incluses. Le système se limitera au suivi des paiements et à la génération des factures.

    \item \textbf{Paiement en ligne :} L'intégration d'un système de paiement en ligne (carte de crédit, virement bancaire) n'est pas prévue. Les paiements continueront à se faire par chèque.

    \item \textbf{Support multilingue :} L'interface sera offerte uniquement en français.
\end{itemize}

%\input{tex/exemples_LaTeX}

%   Annexes
\appendix


\end{document}
% Fin du document
