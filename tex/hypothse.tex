\subsection{Hypothèses, contraintes et risques}

\subsubsection*{Hypothèses}

\begin{itemize}
    \item Les membres disposent d'un accès à Internet et peuvent utiliser un navigateur web ou une application simple.
    
    \item Mme Lepage et les responsables d'équipes seront disponibles pour l'analyse des besoins, la validation et les tests.
    
    \item Les processus actuels documentés par Mme Lepage (Annexe 2) sont suffisamment stables pour servir de base à l'informatisation.
    
    \item BonProBio maintiendra son mode de fonctionnement actuel (formats des catalogues, bordereaux de commande et livraisons) pendant la durée du projet.
    
    \item Un super-utilisateur sera formé pour former ensuite les autres bénévoles (approche « train the trainer »).
    
    \item Les cas limites et exceptions rares pourront être traités manuellement, sans automatisation complète.
    
    \item Mme Lepage acquérera le matériel et les licences nécessaires dès que l'équipe de projet en fera la demande.
\end{itemize}

\subsubsection*{Contraintes}

\begin{itemize}
    \item \textbf{Contrainte budgétaire :} Le groupe d'achat dispose de ressources financières limitées. Le projet doit réduire au minimum les coûts d'acquisition de logiciels et de matériel.
    
    \item \textbf{Contrainte temporelle :} Mme Lepage souhaite un déploiement dans un délai de 18 mois.
    
    \item \textbf{Contrainte technique - Plateforme :} Le logiciel doit fonctionner sous Windows et nécessiter le moins de logiciels supplémentaires possible (logiciels bureautiques et courriel standards).
    
    \item \textbf{Contrainte technique - Infrastructure :} Le groupe ne dispose pas de ressources informatiques propres. Les utilisateurs emploieront leurs ordinateurs personnels.
    
    \item \textbf{Contrainte d'ergonomie :} L'interface doit être très simple et adaptée à des utilisateurs ayant peu d'habiletés informatiques.
    
    \item \textbf{Contrainte d'installation :} Les procédures d'installation doivent être très simples.
    
    \item \textbf{Contrainte de formation :} La gestion du changement, la documentation et la formation seront prises en charge par Mme Lepage. L'équipe de projet doit seulement former un super-utilisateur.
    
    \item \textbf{Contrainte de connectivité :} Tous les utilisateurs disposent d'une connexion Internet, ce qui est un prérequis pour l'utilisation du système.
\end{itemize}

\subsubsection*{Risques}

\begin{table}[H]
\centering
\scriptsize
\setlength{\tabcolsep}{5pt}
\renewcommand{\arraystretch}{1.543}
\begin{tabularx}{\textwidth}{|p{3.2cm}|c|c|X|}
\hline
\textbf{Risque} & \textbf{Probabilité} & \textbf{Impact} & \textbf{Mesure d'atténuation} \\
\hline
Résistance au changement des bénévoles face au nouveau système & Moyenne & Élevé & Impliquer les utilisateurs dès la conception, offrir une formation adaptée et un accompagnement \\
\hline
Faible adoption du système par les membres peu familiers avec l'informatique & Élevée & Élevé & Concevoir une interface très simple et intuitive, prévoir un mode d'emploi visuel \\
\hline
Indisponibilité de Mme Lepage pour valider les livrables & Moyenne & Élevé & Identifier un remplaçant pour les validations, planifier les rencontres à l'avance \\
\hline
Changement des processus de BonProBio en cours de projet & Faible & Moyen & Concevoir un système flexible et paramétrable \\
\hline
Dépassement de l'échéancier de 18 mois & Moyenne & Moyen & Planification rigoureuse, suivi régulier de l'avancement, priorisation des fonctionnalités essentielles \\
\hline
Problèmes techniques liés à la diversité des équipements des utilisateurs & Moyenne & Moyen & Privilégier une solution web accessible depuis n'importe quel navigateur \\
\hline
Perte de données ou problèmes de sécurité & Faible & Élevé & Mettre en place des sauvegardes régulières et des mesures de sécurité appropriées \\
\hline
\end{tabularx}
\caption{Registre des risques du projet SGI-Yakabio}
\end{table}
