\subsection{Hypothèses, contraintes et risques}

\subsubsection*{Hypothèses}

\begin{itemize}
    \item Les membres ont accès à Internet et sont en mesure d'utiliser un navigateur web ou une application simple.

    \item Mme Lepage et les responsables d'équipes seront disponibles pour participer à l'analyse des besoins, valider les livrables et effectuer les tests.

    \item Les processus actuels, tels que documentés par Mme Lepage (Annexe 2), sont suffisamment stables pour servir de base à l'informatisation.

    \item BonProBio conservera son mode de fonctionnement actuel (formats des catalogues, bordereaux de commande et livraisons) pendant toute la durée du projet.

    \item Un super-utilisateur sera formé et se chargera ensuite de former les autres bénévoles (approche « former le formateur »).

    \item Les cas particuliers et les exceptions rares pourront être traités manuellement, sans qu'il soit nécessaire de les automatiser.

    \item Mme Lepage se procurera le matériel et les licences nécessaires dès que l'équipe de projet en fera la demande.
\end{itemize}
\newpage

\subsubsection*{Contraintes}

\begin{itemize}
    \item \textbf{Contrainte budgétaire :} Le groupe d'achat dispose de ressources financières limitées. Le projet doit donc minimiser les coûts liés à l'acquisition de logiciels et de matériel.

    \item \textbf{Contrainte temporelle :} Mme Lepage souhaite que le système soit déployé dans un délai maximal de 18 mois.

    \item \textbf{Contrainte technique -- Plateforme :} Le logiciel doit fonctionner sous Windows et exiger le moins de logiciels supplémentaires possible (uniquement des logiciels bureautiques et un client courriel standards).

    \item \textbf{Contrainte technique -- Infrastructure :} Le groupe ne possède pas de ressources informatiques propres. Les utilisateurs devront se servir de leurs ordinateurs personnels.

    \item \textbf{Contrainte d'ergonomie :} L'interface doit être très simple et adaptée à des utilisateurs qui ne sont pas à l'aise avec l'informatique.

    \item \textbf{Contrainte d'installation :} Les procédures d'installation doivent être aussi simples que possible.

    \item \textbf{Contrainte de formation :} La gestion du changement, la documentation et la formation des bénévoles seront prises en charge par Mme Lepage. L'équipe de projet n'aura qu'à former un super-utilisateur.

    \item \textbf{Contrainte de connectivité :} Tous les utilisateurs doivent disposer d'une connexion Internet, car il s'agit d'un prérequis pour utiliser le système.
\end{itemize}

\subsubsection*{Risques}

\begin{table}[H]
\centering
\scriptsize
\setlength{\tabcolsep}{6pt}
\renewcommand{\arraystretch}{1.73}
\captionsetup{justification=centering}
\begin{tabularx}{\textwidth}{|p{4cm}|c|c|X|}
\hline
\rowcolor{headerblue}
\textcolor{white}{\textbf{Risque}} & \textcolor{white}{\textbf{Probabilité}} & \textcolor{white}{\textbf{Impact}} & \textcolor{white}{\textbf{Mesure d'atténuation}} \\
\hline
\rowcolor{lightgray}
Résistance des bénévoles face au changement et au nouveau système & Moyenne & Élevé & Impliquer les utilisateurs dès la phase de conception, leur offrir une formation adaptée et un accompagnement personnalisé \\
\hline
Faible adoption du système par les membres peu à l'aise avec l'informatique & Élevée & Élevé & Concevoir une interface très simple et intuitive, accompagnée d'un mode d'emploi visuel \\
\hline
\rowcolor{lightgray}
Indisponibilité de Mme Lepage pour valider les livrables & Moyenne & Élevé & Identifier une personne remplaçante pour les validations et planifier les rencontres bien à l'avance \\
\hline
Changement des processus de BonProBio en cours de projet & Faible & Moyen & Concevoir un système flexible et facilement paramétrable \\
\hline
\rowcolor{lightgray}
Dépassement de l'échéancier de 18 mois & Moyenne & Moyen & Planifier rigoureusement, assurer un suivi régulier de l'avancement et prioriser les fonctionnalités essentielles \\
\hline
Problèmes techniques liés à la diversité des équipements des utilisateurs & Moyenne & Moyen & Privilégier une solution web accessible depuis n'importe quel navigateur standard \\
\hline
\rowcolor{lightgray}
Perte de données ou problèmes de sécurité & Faible & Élevé & Mettre en place des sauvegardes régulières et des mesures de sécurité appropriées \\
\hline
\end{tabularx}
\caption{Registre des risques du projet SGI-Yakabio}
\end{table}
