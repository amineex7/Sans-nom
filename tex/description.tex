\subsection{Description de la situation actuelle, des besoins et des problématiques}

\subsubsection*{Situation actuelle}

Le groupe d'achat Yakabio fonctionne actuellement avec une organisation répartie en cinq équipes de bénévoles. Chaque mois, les membres soumettent leurs commandes individuelles, qui sont ensuite compilées et transmises à la coopérative BonProBio. Les produits sont livrés au centre communautaire CommunoCentre, où ils sont réceptionnés, triés puis distribués aux membres. Les paiements s'effectuent par chèque : la coordonnatrice dépose les chèques reçus à la caisse populaire, puis émet un chèque global destiné à BonProBio.

L'ensemble des opérations repose en grande partie sur des tâches manuelles, des documents papier et des échanges par courriel ou téléphone. La coordination des bénévoles est principalement assurée par Mme Lepage, qui est actuellement la seule personne à avoir une vision globale du fonctionnement du groupe.

\subsubsection*{Problématiques}

\begin{itemize}
    \item \textbf{Désistement des responsables d'équipes :} Il est difficile de trouver des personnes prêtes à assumer un rôle de coordination. Lorsqu'un responsable quitte, son remplacement et la formation de son successeur demandent beaucoup de temps.

    \item \textbf{Désistement des membres d'équipes :} Les bénévoles ont des obligations familiales et professionnelles qui limitent leur disponibilité. Les désistements surviennent souvent à la dernière minute, sans préavis.

    \item \textbf{Manque d'organisation des coordonnateurs d'équipes :} Plusieurs responsables tardent à planifier leurs activités et à contacter les membres de leur équipe, ce qui entraîne des problèmes de dernière minute.

    \item \textbf{Usage irrégulier du courriel :} Plusieurs membres ne consultent pas régulièrement leurs courriels, ce qui complique les communications et la coordination.

    \item \textbf{Concentration des connaissances :} Mme Lepage est la seule à bien comprendre l'ensemble du fonctionnement du groupe. Si elle devait s'absenter, le groupe se retrouverait rapidement en difficulté.

    \item \textbf{Processus manuels et risques d'erreurs :} La répartition des frais de livraison, la vérification des bordereaux de commande et la gestion des paiements se font manuellement, ce qui augmente considérablement le risque d'erreurs.

    \item \textbf{Manque d'outils de suivi et de rappel :} Les bénévoles ne disposent d'aucun outil pour leur rappeler les tâches à accomplir et les échéances à respecter, ce qui contribue aux oublis et aux désistements.
\end{itemize}

\subsubsection*{Besoins}

\begin{itemize}
    \item \textbf{Automatisation des commandes :} Un système qui permet aux membres de soumettre leurs commandes en ligne et qui génère automatiquement la commande groupée à envoyer au fournisseur.

    \item \textbf{Gestion automatisée de la facturation :} Un calcul automatique des montants, des taxes et des frais de livraison pour chaque membre, afin d'éliminer les erreurs de calcul.

    \item \textbf{Suivi des livraisons :} Un module permettant de savoir en temps réel quels colis ont été reçus et distribués aux membres.

    \item \textbf{Coordination des équipes :} Un outil qui facilite la gestion des disponibilités, l'assignation des tâches et l'envoi de rappels automatiques aux bénévoles.

    \item \textbf{Gestion des membres :} Une base de données centralisée regroupant les coordonnées des membres, leur appartenance aux équipes et leur historique de participation.

    \item \textbf{Support multi-fournisseurs :} La possibilité de gérer des commandes auprès de plusieurs fournisseurs, chacun avec son propre calendrier.

    \item \textbf{Simplicité d'utilisation :} Une interface simple et intuitive, adaptée à des utilisateurs qui ne sont pas nécessairement à l'aise avec l'informatique.

    \item \textbf{Traçabilité et visibilité :} Des rapports clairs permettant de suivre les volumes commandés, les coûts et les habitudes de consommation, pour faciliter la prise de décision.
\end{itemize}
