\subsection{Description de la situation actuelle, des besoins et des problématiques}

\subsubsection*{Situation actuelle}

Le groupe d'achat Yakabio fonctionne selon une organisation en cinq équipes de bénévoles. Chaque mois, les membres soumettent leurs commandes individuelles, qui sont compilées puis transmises à la coopérative BonProBio. Les produits sont livrés au centre communautaire CommunoCentre, où ils sont réceptionnés, triés et distribués aux membres. Les paiements se font par chèque : la coordonnatrice dépose les chèques à la caisse populaire et émet ensuite un chèque global à BonProBio.

L'ensemble des processus repose largement sur des tâches manuelles, des documents papier et des communications par courriel ou téléphone. La coordination des bénévoles est assurée principalement par Mme Lepage, qui détient à elle seule une vision globale du fonctionnement du groupe.

\subsubsection*{Problématiques}

\begin{itemize}
    \item \textbf{Désistement des responsables d'équipes :} Il est difficile d'identifier des personnes capables d'assumer la coordination. Lorsqu'un responsable quitte, le remplacement et la formation prennent beaucoup de temps.
    
    \item \textbf{Désistement des membres d'équipes :} Les bénévoles ont des contraintes familiales et professionnelles qui limitent leur disponibilité. Les désistements surviennent souvent à la dernière minute, sans préavis.
    
    \item \textbf{Manque d'organisation des coordonnateurs d'équipes :} Plusieurs responsables ne planifient pas assez tôt et ne contactent pas les membres de leur équipe à temps, ce qui entraîne des problèmes de dernière minute.
    
    \item \textbf{Usage irrégulier du courriel :} Plusieurs membres ne consultent pas régulièrement leur courrier électronique, ce qui complique la communication et la coordination.
    
    \item \textbf{Concentration des connaissances :} Mme Lepage est la seule à avoir une compréhension globale du fonctionnement du groupe. En cas d'absence, le groupe serait fortement fragilisé.
    
    \item \textbf{Processus manuels et sources d'erreurs :} La répartition des frais de livraison, la vérification des bordereaux de commande et la gestion des paiements sont effectuées manuellement, ce qui accroît le risque d'erreurs.
    
    \item \textbf{Manque d'outils de suivi et de rappel :} Les bénévoles ne disposent pas d'outils pour rappeler les tâches et les échéances, ce qui contribue aux oublis et aux désistements.
\end{itemize}

\subsubsection*{Besoins}

\begin{itemize}
    \item \textbf{Automatisation des commandes :} Un système permettant aux membres de soumettre leurs commandes en ligne et de générer automatiquement la commande groupée.
    
    \item \textbf{Gestion automatisée de la facturation :} Calcul automatique des montants, des taxes et des frais de livraison pour chaque membre.
    
    \item \textbf{Suivi des livraisons :} Un module de suivi permettant de savoir quels colis ont été reçus et distribués.
    
    \item \textbf{Coordination des équipes :} Un outil permettant de gérer les disponibilités, d'assigner les tâches et d'envoyer des rappels automatiques.
    
    \item \textbf{Gestion du membership :} Une base de données centralisée avec les coordonnées, l'appartenance aux équipes et l'historique de participation des membres.
    
    \item \textbf{Support multi-fournisseurs :} La possibilité de gérer des commandes auprès de plusieurs fournisseurs avec des calendriers distincts.
    
    \item \textbf{Simplicité d'utilisation :} Une interface simple et intuitive, adaptée à des utilisateurs ayant peu d'habiletés informatiques.
    
    \item \textbf{Traçabilité et visibilité :} Des rapports permettant de suivre les volumes, les coûts et les habitudes de consommation afin de faciliter la prise de décision.
\end{itemize}
