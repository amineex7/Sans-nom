%!TEX encoding = UTF-8 Unicode

%
% Sections "Impacts", "Bénéfices" et "Intervenants"
%

\subsection{Impacts du projet}

La mise en place du système SGI-Yakabio aura plusieurs impacts sur le fonctionnement du groupe d'achat et sur ses membres.

\subsubsection*{Impacts organisationnels}

\begin{itemize}
    \item \textbf{Changement des habitudes de travail :} Les bénévoles devront s'adapter à de nouvelles façons de faire. Les tâches qui étaient auparavant effectuées manuellement seront désormais réalisées à l'aide du système informatique.

    \item \textbf{Nouvelle répartition des responsabilités :} Le système permettra une meilleure distribution des tâches entre les équipes, ce qui réduira la charge de travail de la coordonnatrice et des responsables d'équipes.

    \item \textbf{Besoin de formation :} Les bénévoles devront être formés à l'utilisation du nouveau système. Un effort d'accompagnement sera nécessaire, surtout pour les membres moins à l'aise avec l'informatique.
\end{itemize}

\subsubsection*{Impacts techniques}

\begin{itemize}
    \item \textbf{Dépendance à la technologie :} Le groupe deviendra dépendant du bon fonctionnement du système informatique et de la connexion Internet. Des procédures de secours devront être prévues en cas de panne.

    \item \textbf{Gestion des données :} Les informations des membres et les historiques de commandes seront stockés de façon centralisée, ce qui nécessitera une attention particulière à la sécurité et à la sauvegarde des données.
\end{itemize}

\subsubsection*{Impacts sur les membres}

\begin{itemize}
    \item \textbf{Adaptation au nouveau processus de commande :} Les membres devront apprendre à soumettre leurs commandes en ligne plutôt que par les moyens traditionnels (papier, courriel).

    \item \textbf{Meilleure visibilité :} Les membres auront un accès plus facile à l'information concernant leurs commandes, leurs paiements et leur historique.
\end{itemize}

\subsection{Bénéfices attendus}

Le projet SGI-Yakabio apportera de nombreux bénéfices au groupe d'achat, tant sur le plan opérationnel que sur le plan humain.

\subsubsection*{Bénéfices opérationnels}

\begin{itemize}
    \item \textbf{Réduction des erreurs :} L'automatisation des calculs (montants, taxes, frais de livraison) éliminera les erreurs de calcul qui surviennent lors du traitement manuel des commandes.

    \item \textbf{Gain de temps :} La compilation automatique des commandes et la génération des documents réduiront considérablement le temps consacré aux tâches administratives répétitives.

    \item \textbf{Meilleure coordination :} Les rappels automatiques et la gestion centralisée des disponibilités faciliteront la coordination entre les équipes et réduiront les désistements de dernière minute.

    \item \textbf{Traçabilité améliorée :} L'historique des commandes et des paiements sera facilement accessible, ce qui simplifiera le suivi et la résolution des problèmes.
\end{itemize}

\subsubsection*{Bénéfices pour les membres}

\begin{itemize}
    \item \textbf{Simplicité du processus de commande :} Les membres pourront passer leurs commandes en ligne à tout moment, sans avoir à se déplacer ou à envoyer des courriels.

    \item \textbf{Transparence :} Chaque membre pourra consulter l'état de ses commandes, ses factures et son historique de participation.

    \item \textbf{Réduction des délais :} Le traitement plus rapide des commandes permettra une meilleure planification et une distribution plus efficace des produits.
\end{itemize}

\subsubsection*{Bénéfices pour la pérennité du groupe}

\begin{itemize}
    \item \textbf{Réduction de la dépendance aux personnes clés :} La documentation des processus dans le système et la centralisation de l'information permettront au groupe de fonctionner même en cas d'absence de la coordonnatrice.

    \item \textbf{Facilité d'intégration des nouveaux bénévoles :} Les nouveaux membres pourront se familiariser plus rapidement avec les opérations grâce à une interface claire et des procédures bien définies.

    \item \textbf{Potentiel de croissance :} Le système pourra accommoder une augmentation du nombre de membres ou l'ajout de nouveaux fournisseurs dans le futur.

    \item \textbf{Possibilité de commercialisation :} Le logiciel pourra être adapté et vendu à d'autres groupes d'achat, ce qui permettrait de rentabiliser l'investissement initial.
\end{itemize}

\subsection{Intervenants du projet}

Cette section présente les différentes personnes et entités impliquées dans le projet, ainsi que leur rôle respectif.

\subsubsection*{Client du projet}

\begin{itemize}
    \item \textbf{Mme Karine Lepage} -- Coordonnatrice du groupe d'achat Yakabio. Elle est la principale interlocutrice pour la définition des besoins, la validation des livrables et l'acceptation finale du système. Elle sera également responsable de la formation des bénévoles et de la gestion du changement au sein du groupe.
\end{itemize}

\subsubsection*{Utilisateurs du système}

\begin{itemize}
    \item \textbf{La coordonnatrice (Mme Lepage)} -- Utilisatrice principale du système avec des droits d'administration complets. Elle aura accès à toutes les fonctionnalités, y compris les rapports et les statistiques.

    \item \textbf{Les responsables d'équipes} -- Utilisateurs ayant des droits de gestion pour leur équipe respective. Ils pourront gérer les disponibilités, assigner les tâches et consulter les informations pertinentes à leur équipe.

    \item \textbf{Les bénévoles} -- Utilisateurs ayant des droits limités leur permettant de consulter les tâches qui leur sont assignées et de confirmer leur disponibilité.

    \item \textbf{Les membres du groupe d'achat} -- Utilisateurs ayant des droits de base pour passer des commandes, consulter leurs factures et mettre à jour leurs informations personnelles.
\end{itemize}

\subsubsection*{Parties prenantes externes}

\begin{itemize}
    \item \textbf{Coopérative BonProBio} -- Fournisseur principal du groupe d'achat. Bien qu'elle ne soit pas directement impliquée dans le projet, le système devra être compatible avec ses formats de catalogues et de bordereaux de commande.

    \item \textbf{Centre communautaire CommunoCentre} -- Lieu de réception et de distribution des commandes. Le système devra tenir compte des contraintes liées à ce lieu (horaires, disponibilité des locaux).
\end{itemize}
