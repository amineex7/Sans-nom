%!TEX encoding = UTF-8 Unicode

%
% Section "Conclusion"
%

\section{Conclusion}
\label{s:conclusion}

Ce rapport nous a permis d'analyser la situation actuelle du groupe d'achat Yakabio dans un contexte de changements organisationnels importants. L'étude du fonctionnement du groupe a mis en évidence plusieurs faiblesses, notamment la forte dépendance à une seule personne, la difficulté de coordination entre les bénévoles et la multiplication des erreurs administratives liées à des processus majoritairement manuels.

La rédaction de la charte de projet a permis de structurer la réflexion autour de ces enjeux en identifiant clairement les problématiques, les besoins, les critères de succès ainsi que les risques et contraintes associés au projet. Cette démarche constitue une étape essentielle afin de poser des bases solides pour la mise en place d'une solution organisationnelle et informatique adaptée aux réalités de Yakabio.

\subsection{Bilan du travail réalisé et faisabilité du projet}

À ce stade du projet, plusieurs éléments clés ont été complétés, notamment l'analyse du contexte, l'identification des problématiques et des besoins, la définition des critères de succès ainsi que la rédaction du compte rendu de réunion. La répartition des tâches au sein de l'équipe a également été effectuée, ce qui permet une progression structurée et organisée du travail.

Du point de vue de la faisabilité, le projet apparaît réaliste et réalisable. Les besoins exprimés par le groupe Yakabio sont clairs et bien définis, et la solution repose sur des outils informatiques simples et accessibles, adaptés à un environnement bénévole. Aucun obstacle majeur n'a été identifié à ce stade. Bien que certaines contraintes et risques existent, notamment liés à la disponibilité des bénévoles et à la gestion du changement, ceux-ci demeurent gérables dans le cadre du projet proposé. Le projet est donc jugé faisable tant sur le plan organisationnel que technique.

\subsection{Recommandations pour la suite du travail}

Afin d'assurer la réussite du projet du groupe d'achat Yakabio, il est recommandé :

\begin{itemize}
    \item D'assurer un suivi de l'échéancier et des responsabilités, compte tenu de la disponibilité des bénévoles et du caractère communautaire de l'organisation.

    \item De valider régulièrement les livrables avec la coordonnatrice du groupe, afin de s'assurer que les besoins exprimés sont bien compris et que les solutions proposées demeurent alignées avec la réalité opérationnelle de Yakabio.

    \item De documenter clairement les processus et les décisions prises, dans le but de réduire la dépendance envers des personnes clés et de faciliter la continuité des opérations lors de changements futurs.

    \item D'impliquer progressivement les bénévoles dans le projet, afin de favoriser l'adhésion aux changements.

    \item De prévoir un plan de formation simple et accessible, permettant aux bénévoles de se familiariser avec les nouveaux outils ou processus sans alourdir leur charge de travail.

    \item De maintenir une communication claire et régulière au sein de l'équipe de projet, en utilisant des outils collaboratifs pour assurer une coordination efficace et un suivi constant de l'avancement.
\end{itemize}
