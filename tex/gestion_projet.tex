%!TEX encoding = UTF-8 Unicode

%
% Section "Gestion de projets"
%

\section{Compte rendu de la gestion de projet}
\label{s:gestion}

Cette section présente l'organisation de notre équipe de travail, les activités réalisées dans le cadre de ce projet, ainsi qu'une réflexion sur notre fonctionnement en tant qu'équipe.

\subsection{Composition de l'équipe et rôles}

L'équipe chargée du projet est constituée de quatre membres. Le tableau suivant présente chacun d'eux ainsi que les principales responsabilités qui leur sont attribuées au sein du groupe.

\begin{table}[H]
\centering
\renewcommand{\arraystretch}{1.4}
\begin{tabularx}{\textwidth}{|l|X|}
\hline
\rowcolor{headerblue}
\textcolor{white}{\textbf{Nom}} & \textcolor{white}{\textbf{Description des tâches}} \\
\hline
\rowcolor{lightgray}
Salma Mouhsin & Chargée de projet, coordination de l'équipe, communication avec le client et rédaction du rapport \\
\hline
Vincent Méroz & Analyse des besoins, rédaction des sections sur les problématiques et critères de succès \\
\hline
\rowcolor{lightgray}
Mohamed Amine Annane & Analyse de la solution, rédaction des sections sur les impacts et la portée du projet \\
\hline
Iréti Rohan Yannis Kurtis Wilfried Zoumenou & Analyse des risques, rédaction des sections sur les hypothèses et la conclusion \\
\hline
\end{tabularx}
\caption{Composition de l'équipe et répartition des tâches}
\end{table}

\subsection{Compte rendu des principales activités réalisées}

\subsubsection*{Réunion 1 -- Discussion du TP et répartition des tâches}

\begin{table}[H]
\centering
\renewcommand{\arraystretch}{1.3}
\begin{tabularx}{\textwidth}{|>{\bfseries}l|X|}
\hline
\cellcolor{headerblue}\textcolor{white}{Date} & 29 janvier 2026 \\
\hline
\rowcolor{lightgray}
Heure & De 19h à 19h30 \\
\hline
\cellcolor{headerblue}\textcolor{white}{Endroit} & Teams (réunion virtuelle) \\
\hline
\rowcolor{lightgray}
Objet & Discussion du TP et répartition des tâches \\
\hline
\cellcolor{headerblue}\textcolor{white}{Participants} & Tous les membres de l'équipe \\
\hline
\end{tabularx}
\end{table}

\textbf{Ordre du jour :}

\begin{table}[H]
\centering
\renewcommand{\arraystretch}{1.3}
\begin{tabularx}{\textwidth}{|X|c|}
\hline
\rowcolor{headerblue}
\textcolor{white}{\textbf{Point à l'ordre du jour}} & \textcolor{white}{\textbf{Durée}} \\
\hline
\rowcolor{lightgray}
1. Ouverture de la réunion & 2 min \\
\hline
2. Présentation du TP1 et des attentes du professeur & 7 min \\
\hline
\rowcolor{lightgray}
3. Discussion sur la structure du rapport et la charte de projet & 10 min \\
\hline
4. Nomination du chargé de projet & 3 min \\
\hline
\rowcolor{lightgray}
5. Planification des prochaines étapes & 8 min \\
\hline
\end{tabularx}
\caption{Ordre du jour de la réunion du 29 janvier 2026}
\end{table}
\newpage

\textbf{Décisions prises lors de cette réunion :}
\begin{itemize}
    \item Salma Mouhsin a été nommée chargée de projet à l'unanimité.
    \item La structure du rapport a été validée selon le modèle fourni par le professeur.
    \item Chaque membre s'est vu attribuer des sections spécifiques à rédiger.
    \item La prochaine réunion a été planifiée pour faire le point sur l'avancement.
\end{itemize}
\begin{comment}
\subsubsection*{Réunion 3 -- Révision finale}

\begin{table}[H]
\centering
\renewcommand{\arraystretch}{1.3}
\begin{tabularx}{\textwidth}{|>{\bfseries}l|X|}
\hline
\cellcolor{headerblue}\textcolor{white}{Date} & 10 février 2026 \\
\hline
\rowcolor{lightgray}
Heure & De 18h à 19h \\
\hline
\cellcolor{headerblue}\textcolor{white}{Endroit} & Teams (réunion virtuelle) \\
\hline
\rowcolor{lightgray}
Objet & Révision finale et validation du rapport \\
\hline
\cellcolor{headerblue}\textcolor{white}{Participants} & Tous les membres de l'équipe \\
\hline
\end{tabularx}
\end{table}

\textbf{Points abordés :}
\begin{itemize}
    \item Relecture complète du rapport par tous les membres.
    \item Correction des erreurs de français et harmonisation du style.
    \item Vérification de la conformité avec le barème d'évaluation.
    \item Validation finale du document avant la remise.
\end{itemize}
\end{comment}
\subsection{Problèmes rencontrés et suggestions de solutions}

Au cours de ce projet, nous avons rencontré quelques difficultés que nous avons su surmonter grâce à une bonne communication au sein de l'équipe.

\subsubsection*{Problèmes rencontrés}

\begin{itemize}
    \item \textbf{Coordination des horaires :} Il a parfois été difficile de trouver des créneaux où tous les membres étaient disponibles pour les réunions d'équipe, compte tenu des emplois du temps chargés de chacun. Nous avons résolu ce problème en privilégiant les réunions en soirée sur Teams.

    \item \textbf{Compréhension du contexte du groupe d'achat :} Au départ, il n'était pas facile de bien comprendre toutes les subtilités du fonctionnement d'un groupe d'achat communautaire. La lecture attentive des documents fournis et les discussions en équipe nous ont permis de clarifier plusieurs points.

    \item \textbf{Harmonisation du style de rédaction :} Comme chaque membre a rédigé des sections différentes, il a fallu un travail de révision pour assurer une cohérence dans le ton et le style du rapport.

    \item \textbf{Gestion du temps :} Avec les autres cours et obligations, il a fallu bien planifier le travail pour respecter l'échéance de remise.
\end{itemize}

\subsubsection*{Suggestions de solutions}

\begin{itemize}
    \item \textbf{Planifier les rencontres dès le début :} Fixer les dates des réunions dès le lancement du projet pour éviter les conflits d'horaire.

    \item \textbf{Établir un guide de style commun :} Définir dès le départ des conventions de rédaction pour faciliter l'harmonisation des contributions de chaque membre.

    \item \textbf{Utiliser des outils collaboratifs :} Continuer à utiliser Teams et d'autres outils de partage de documents pour faciliter la collaboration à distance.

    \item \textbf{Prévoir des marges de temps :} Intégrer des périodes tampons dans l'échéancier pour faire face aux imprévus.
\end{itemize}

\subsection{Évaluation globale de la gestion de projet et d'équipe}

Dans l'ensemble, notre équipe a bien fonctionné tout au long de ce projet. La communication entre les membres a été bonne, et chacun a respecté ses engagements en termes de délais et de qualité du travail.

\subsubsection*{Points forts de l'équipe}

\begin{itemize}
    \item \textbf{Bonne répartition des tâches :} Dès la première réunion, chaque membre s'est vu attribuer des responsabilités claires, ce qui a évité les chevauchements et les oublis.

    \item \textbf{Communication efficace :} L'utilisation de Teams a permis de maintenir un contact régulier entre les membres, même en dehors des réunions formelles.

    \item \textbf{Entraide :} Lorsqu'un membre avait besoin d'aide ou de clarification, les autres étaient disponibles pour l'appuyer.

    \item \textbf{Respect des échéances :} Toutes les étapes du projet ont été réalisées dans les délais prévus, ce qui nous a permis de remettre le rapport à temps.
\end{itemize}

\subsubsection*{Points à améliorer}

\begin{itemize}
    \item \textbf{Réunions plus fréquentes :} Des points de suivi plus réguliers auraient permis de détecter plus tôt certains problèmes de coordination.

\end{itemize}

\subsubsection*{Conclusion}

Ce projet a été une expérience enrichissante qui nous a permis de développer nos compétences en analyse de besoins, en rédaction technique et en travail collaboratif. La réalisation de la charte de projet nous a donné une meilleure compréhension des enjeux liés à la gestion d'un projet informatique dans un contexte communautaire. Nous sommes satisfaits du résultat obtenu et confiants que ce rapport constitue une base solide pour la suite du projet SGI-Yakabio.
