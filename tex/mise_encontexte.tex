\subsection{Mise en contexte de la demande}

Le groupe d'achat Yakabio est affilié à la coopérative BonProBio, située près de Lévis, qui approvisionne en produits biologiques les magasins de produits naturels des régions de Québec et de la Beauce. Yakabio rassemble actuellement 72 membres d'un quartier de la ville de Québec, qui se regroupent pour passer des commandes collectives et ainsi bénéficier des prix de gros offerts par la coopérative.

Depuis sa création, le groupe fonctionnait efficacement grâce à l'implication de M. Jean Després, un retraité bénévole qui assumait à lui seul l'ensemble des tâches de coordination : recevoir les commandes individuelles, compiler et transmettre la commande groupée à BonProBio, réceptionner et distribuer les colis, gérer les paiements et communiquer avec les membres. En 2024, M. Després a reçu un diagnostic de leucémie, ce qui l'a amené à envisager un retrait progressif de ses responsabilités. \newpage

En février 2025, il a définitivement quitté son rôle. Mme Karine Lepage a alors pris le relais en mettant en place une nouvelle organisation reposant sur cinq équipes de bénévoles : Commande, Livraison, Distribution, Paiement et Coordination. Malgré cette réorganisation, des difficultés sont rapidement apparues dès la rentrée de septembre 2025, notamment des désistements de bénévoles et des problèmes de coordination entre les équipes.

Face à ces défis, Mme Lepage souhaite mener une étude approfondie du fonctionnement actuel et mettre en place un système informatisé pour optimiser les opérations. Elle envisage également, à plus long terme, de commercialiser ce logiciel auprès d'autres groupes d'achat afin de rentabiliser l'investissement.
