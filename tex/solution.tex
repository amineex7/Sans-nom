\subsection{Solution proposée et portée du projet}

\subsubsection*{Solution proposée}

La solution proposée consiste à développer un système informatisé de gestion intégré pour le groupe d'achat Yakabio (SGI-Yakabio). Il s'agira d'une application web accessible au moyen d'un navigateur standard, qui permettra aux membres et aux responsables d'équipes d'effectuer leurs tâches de manière centralisée et automatisée.

Le système sera conçu de manière modulaire et paramétrable afin de pouvoir être adapté et commercialisé auprès d'autres groupes d'achat. Il fonctionnera sous Windows et ne nécessitera que des logiciels standards (navigateur web, client courriel).

\subsubsection*{Portée du projet}

Le projet comprend les éléments suivants :

\textbf{Module de gestion des membres :}
\begin{itemize}
    \item Inscription et mise à jour des informations des membres (nom, adresse, téléphone, courriel)
    \item Gestion des cotisations annuelles
    \item Attribution des membres aux équipes
    \item Génération et envoi des cartes de membre
\end{itemize}

\textbf{Module de gestion des fournisseurs et catalogues :}
\begin{itemize}
    \item Enregistrement des fournisseurs et de leurs coordonnées
    \item Importation et gestion des catalogues de produits (code, description, prix, taxes)
    \item Support de plusieurs fournisseurs avec des calendriers distincts
\end{itemize}

\textbf{Module de gestion des commandes :}
\begin{itemize}
    \item Interface de saisie des commandes individuelles par les membres
    \item Compilation automatique de la commande groupée
    \item Génération des bordereaux de commande au format requis par les fournisseurs
    \item Calcul automatique des montants, des taxes et des totaux
    \item Suivi du statut des commandes
\end{itemize}

\textbf{Module de gestion des livraisons :}
\begin{itemize}
    \item Enregistrement des livraisons reçues
    \item Calcul et répartition automatique des frais de livraison entre les membres
    \item Génération des factures individuelles pour chaque membre
    \item Suivi de la distribution des colis
\end{itemize}

\textbf{Module de gestion des paiements :}
\begin{itemize}
    \item Enregistrement des paiements reçus des membres
    \item Suivi des chèques et des dépôts bancaires
    \item Génération du sommaire de commande et du chèque au fournisseur
    \item Historique des transactions financières
\end{itemize}

\textbf{Module de coordination des équipes :}
\begin{itemize}
    \item Gestion des équipes et de leurs membres
    \item Saisie des disponibilités des bénévoles
    \item Attribution des tâches pour chaque cycle de commande
    \item Envoi de rappels automatiques (courriel)
    \item Registre de participation des membres aux activités
\end{itemize}

\textbf{Module de paramétrage :}
\begin{itemize}
    \item Configuration des informations du groupe d'achat (nom, adresse, coordonnées)
    \item Personnalisation des documents générés (entêtes, logos)
    \item Gestion du calendrier des commandes
\end{itemize}

\textbf{Module de rapports et statistiques :}
\begin{itemize}
    \item Rapports sur les volumes de commandes
    \item Statistiques sur la participation des membres
    \item Historique des commandes par membre
    \item Tableau de bord pour la coordonnatrice
\end{itemize}

\textbf{Livrables du projet :}
\begin{itemize}
    \item Application web fonctionnelle SGI-Yakabio
    \item Documentation technique du système
    \item Guide d'utilisation pour les utilisateurs
    \item Formation d'un super-utilisateur
    \item Code source et scripts d'installation
\end{itemize}
