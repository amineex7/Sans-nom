\subsection{Critères de succès}

Les critères suivants permettront de déterminer si le projet est un succès à sa conclusion :

\begin{enumerate}
    \item \textbf{Gestion des commandes opérationnelle :} Le système permet aux membres de soumettre leurs commandes en ligne et génère automatiquement la commande groupée à transmettre au fournisseur, sans compilation manuelle.
    
    \item \textbf{Automatisation de la facturation :} Le système calcule automatiquement les montants, les taxes applicables et la répartition des frais de livraison, ce qui réduit les calculs manuels et les erreurs.
    
    \item \textbf{Coordination des équipes facilitée :} Le système permet de gérer les disponibilités, d'assigner les tâches par cycle de commande et d'envoyer des rappels automatiques, ce qui réduit les désistements de dernière minute.
    
    \item \textbf{Gestion centralisée du membership :} Les informations des membres (coordonnées, appartenance aux équipes, cotisations, historique de participation) sont centralisées dans une base de données accessible et à jour.
    
    \item \textbf{Support multi-fournisseurs :} Le système permet de gérer des commandes auprès de plusieurs fournisseurs avec des catalogues et des calendriers distincts.
    
    \item \textbf{Traçabilité complète :} Le système offre une visibilité sur l'historique des commandes, les volumes, les coûts et les habitudes de consommation.
    
    \item \textbf{Facilité d'utilisation :} Les utilisateurs peuvent utiliser le système sans formation extensive. L'interface est intuitive et adaptée à des personnes ayant peu d'habiletés informatiques.
    
    \item \textbf{Déploiement dans les délais :} Le système est livré et opérationnel dans un délai de 18 mois, conformément aux attentes de Mme Lepage.
    
    \item \textbf{Paramétrable et réutilisable :} Le logiciel est suffisamment générique pour être configuré et utilisé par d'autres groupes d'achat, en vue d'une éventuelle commercialisation.
    
    \item \textbf{Réduction de la charge administrative :} Les responsables d'équipes et la coordonnatrice constatent une diminution significative du temps consacré aux tâches administratives.
    
    \item \textbf{Autonomie du groupe :} Le groupe peut fonctionner de manière pérenne sans dépendre d'une seule personne détenant toutes les connaissances.
\end{enumerate}
