\subsection{Critères de succès}

Pour que le projet soit considéré comme un succès, les critères suivants devront être atteints :

\begin{enumerate}
    \item \textbf{Gestion des commandes opérationnelle :} Le système permet aux membres de soumettre leurs commandes en ligne et génère automatiquement la commande groupée à transmettre au fournisseur, sans qu'aucune compilation manuelle ne soit nécessaire.

    \item \textbf{Automatisation de la facturation :} Le système calcule automatiquement les montants, les taxes applicables et la répartition des frais de livraison, ce qui réduit considérablement les calculs manuels et les risques d'erreurs.

    \item \textbf{Coordination des équipes facilitée :} Le système permet de gérer les disponibilités des bénévoles, d'assigner les tâches pour chaque cycle de commande et d'envoyer des rappels automatiques, ce qui diminue les désistements de dernière minute.

    \item \textbf{Gestion centralisée des membres :} Toutes les informations des membres (coordonnées, appartenance aux équipes, cotisations, historique de participation) sont regroupées dans une base de données accessible et tenue à jour.

    \item \textbf{Support multi-fournisseurs :} Le système permet de gérer des commandes auprès de plusieurs fournisseurs, chacun avec son propre catalogue et son propre calendrier.

    \item \textbf{Traçabilité complète :} Le système offre une vue d'ensemble sur l'historique des commandes, les volumes commandés, les coûts et les habitudes de consommation des membres.

    \item \textbf{Facilité d'utilisation :} Les utilisateurs peuvent se servir du système sans avoir besoin d'une formation poussée. L'interface est intuitive et adaptée à des personnes qui ne sont pas à l'aise avec l'informatique.

    \item \textbf{Déploiement dans les délais :} Le système est livré et opérationnel dans un délai maximal de 18 mois, conformément aux attentes de Mme Lepage.

    \item \textbf{Solution paramétrable et réutilisable :} Le logiciel est suffisamment flexible pour être configuré et utilisé par d'autres groupes d'achat, dans l'optique d'une éventuelle commercialisation.

    \item \textbf{Réduction de la charge administrative :} Les responsables d'équipes et la coordonnatrice constatent une diminution notable du temps qu'ils consacrent aux tâches administratives.

    \item \textbf{Autonomie du groupe :} Le groupe peut fonctionner de façon pérenne sans dépendre d'une seule personne qui détiendrait toutes les connaissances.
\end{enumerate}
